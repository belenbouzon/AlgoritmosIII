\section{Problema 2: A medias}

\subsection{Descripción de la problemática}
En este problema se nos pide calcular todas las medianas parciales, que se obtienen al tomar subconjuntos del original de la siguiente manera:
Suponiendo que tengo el siguiente conjunto $\{2,8,5,3,10\}$, tomamos los subconjuntos
\begin{center}
	$\{2\}$ $\{2,8\}$ $\{2,8,5\}$ $\{2,8,5,3\}$ $\{2,8,5,3,10\}$
\end{center}
Luego calculamos las medianas para cada uno de esos subconjuntos, teniendo en cuenta que deben estar ordenados para poder calcular las medianas correctamente. De esta manera, con el algoritmo que analizaremos a continuación, podemos obtener el siguiente conjunto de medianas $\{2,5,5,4,5\}$ que es solución al problema y que cumple que el $i$-ésimo, con $i = \{0 $..$ 4\}$, representa la parte entera de la mediana de los primeros $i$ números de la entrada. 
\subsection{Resolución propuesta y justificación}

\subsection{Análisis de la complejidad}

\subsection{Código fuente}

\subsection{Experimentación}

\subsubsection{Constrastación Empírica de la complejidad}

