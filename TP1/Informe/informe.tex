\documentclass[a4paper]{article}
\usepackage[spanish]{babel}
\usepackage[utf8]{inputenc}
\usepackage{charter}   % tipografia
\usepackage{graphicx}


%%%%%%%LO AGREGUE%%%%%%%%%%  Y yo lo modifique
%\usepackage{hyperref}
%%%%%%%%%%%%%%%%%%%%%%%%%%

\usepackage[bookmarks = true, colorlinks=true, linkcolor = black, citecolor = black, menucolor = black, urlcolor = blue]{hyperref} 




%\usepackage{makeidx}
\usepackage{paralist} %itemize inline
\usepackage[ruled,vlined]{algorithm2e}
%\usepackage{float}
%\usepackage{amsmath, amsthm, amssymb}
%\usepackage{amsfonts}
%\usepackage{sectsty}
%\usepackage{charter}
%\usepackage{wrapfig}
%\usepackage{listings}
%\lstset{language=C}


\input{codesnippet}
\input{page.layout}
% \setcounter{secnumdepth}{2}
\usepackage{underscore}
\usepackage{caratula}
%\usepackage{url}
\usepackage{hyperref}

% ******************************************************** %
%              TEMPLATE DE INFORME ORGA2 v0.1              %
% ******************************************************** %
% ******************************************************** %
%                                                          %
% ALGUNOS PAQUETES REQUERIDOS (EN UBUNTU):                 %
% ========================================
%                                                          %
% texlive-latex-base                                       %
% texlive-latex-recommended                                %
% texlive-fonts-recommended                                %
% texlive-latex-extra?                                     %
% texlive-lang-spanish (en ubuntu 13.10)                   %
% texlive-science										  %
% ******************************************************** %



\begin{document}


\thispagestyle{empty}
\materia{Algoritmos y Estructuras de Datos III}
\submateria{Primer Cuatrimestre de 2015}
\titulo{Trabajo Práctico I}
%\subtitulo{subtitulo del trabajo}
\integrante{Aldasoro Agustina}{86/13}{agusaldasoro@gmail.com}
\integrante{Noriega Francisco}{660/12}{frannoriega.92@gmail.com}
\integrante{Zimenspitz Ezequiel}{155/13}{ezeqzim@gmail.com}
\integrante{Zuker Brian}{441/13}{brianzuker@gmail.com}


\maketitle
\newpage

\thispagestyle{empty}
\vfill
\begin{abstract}
Habi\'endonos sido dado una serie de tres problem\'aticas a resolver, se plantean sus respectivas soluciones acorde a los requisitos pedidos. Se adjunta una descripci\'on de cada problema y su soluci\'on, conjunto a su an\'alisis de correctitud y de complejidad sumado a su experimentaci\'on. El lenguaje elegido para llevar a cabo el trabajo es C++.

Dentro de cada \emph{.cpp} est\'a el comando para compilar cada ejercicio desde la carpeta donde se encuentran los mismos. A continuaci\'on se los adjunta. El flag \emph{-std=c++11} debi\'o ser a\~nadido  dado que utilizamos la librer\'ia $<$chrono$>$, la cual nos permiti\'o medir tiempos de ejecuci\'on:
\begin{enumerate}
\item \emph{g++ -o main Zombieland.cpp -std=c++11}

\item \emph{g++ -o main AltaFrecuencia.cpp -std=c++11}

\item \emph{g++ -o main SenorCaballos.cpp -std=c++11}
	
\end{enumerate}



\end{abstract}

\thispagestyle{empty}
\vspace{3cm}
\tableofcontents
\newpage


%\normalsize
\newpage
\input{Zombieland.tex}
\newpage
\input{Altafrecuencia.tex}
\newpage
\input{Senorcaballos.tex}
\newpage

% \section{Objetivos generales}

% El objetivo de este Trabajo Práctico es ...


% \section{Contexto}

% \begin{figure}
%   \begin{center}
% 	\includegraphics[scale=0.66]{imagenes/logouba.jpg}
% 	\caption{Descripcion de la figura}
% 	\label{nombreparareferenciar}
%   \end{center}
% \end{figure}


% \paragraph{\textbf{Titulo del parrafo} } Bla bla bla bla.
% Esto se muestra en la figura~\ref{nombreparareferenciar}.




%Habra que insertar el enunciado???
% %\section{Enunciado y solucion} 
% %\input{enunciado}

% \section{Conclusiones y trabajo futuro}

\end{document}

