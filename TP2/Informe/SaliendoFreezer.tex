\section{Problema 1: Saliendo del Freezer}

\subsection{Descripción de la problemática}
En este ejercicio se nos presenta una edificación de N niveles cuyo flujo de transporte entre distintos pisos - no necesariamente consecutivos - se encuentra dado por teletransportación a través de portales unidireccionales (es decir, que sólo permiten el ascenso).\\
A partir de este contexto y asegurando que existe al menos una solución para cada instancia, se nos pide dise\~nar un algoritmo de complejidad O($n^{2}$) que calcule la mayor cantidad de portales que pueden ser usados para subir desde planta baja al piso N, sin descender en ningún momento.\\

\subsection{Resolución propuesta y justificación}
Para garantizar la cota superior de complejidad solicitada propoponemos resolver el problema utilizando un algoritmo de programación dinámica cuyo comportamiento esté regido por las siguientes instrucciones: \\

\begin{itemize}
\item tomar un piso (partiendo por el superior)
\item para cada uno, fijarse a qué otros niveles superiores comunica.
\item para cada uno de ellos comparar la máxima cantidad de portales que los separa del último nivel
\item tomar el máximo de ellos y almacenarlo 
\item al llegar al primer piso, devolver el máximo calculado.
\end{itemize}

este comportamiento puede organizarse y entenderse a través del siguiente pseudocódigo:\\
\textcolor{blue}{completar una vez qe tenga el algoritmo}
\begin{algorithmic} 
	\STATE bla
\end{algorithmic} 


\subsection{Análisis de la complejidad}

\subsection{Código fuente}

\subsection{Experimentación}

\subsubsection{Constrastación Empírica de la complejidad}