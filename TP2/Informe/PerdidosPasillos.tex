\section{Problema 3: Perdidos en los Pasillos}

\subsection{Descripción de la problemática}

En este ejercicio se nos presenta un pabellón con M pasillos de -potencialmente- distintas longitudes y un conjunto vértices que pueden ser tanto intersecciones en las cuales dos o más de ellos convergen, o extremos incididos por un sólo corredor. 
A partir de este contexto se nos pide desarrollar un algoritmo que elimine cualquier ciclo posible del grafo dado, logrando crear, partiendo del mismo, un árbol generador cuyo peso (dado por la sumatoria de los pesos de cada arista) sea mayor o igual al de cualquier otro árbol generador posible (conocido como árbol generador máximo).

\subsection{Resolución propuesta y justificación}

Para resolver el problema, desarrollamos un método que hace uso de una adaptación del algoritmo de Kruskal, mediante el cuál es posible encontrar un árbol generador mínimo con una complejidad de O(m log m) (siendo m la cantidad de aristas).\\
El algoritmo propuesto ordena inicialmente cada pasillo de acuerdo a sus longitudes y los toma uno a uno de mayor a menor verificando si conectan vértices entre los cuales ya existe un camino o no. De ser cierto, la arista en cuestión es descartada. De ser falso, la misma pasa a ser parte del conjunto solución y se repite el procedimiento con la siguiente hasta que se hayan analizado todas.\\
Para que las complejidades se ajustaran a los requerimientos, fue necesario desarrollar la clase \textit{UnionFind}, la cual implementa los métodos \textit{findSet}, \textit{unionSet} e \textit{isSameSet} . El primero de ellos permite, dado un elemento, hallar al representante del conjunto en el que se encuentra. El segundo, por su parte, realiza la unión de dos conjuntos. Por último, el tercero analiza si los dos valores parametrizados están incluídos en el mismo conjunto. \\

El pseudocódigo de nuestro algoritmo es el siguiente:

\begin{algorithmic} 

\STATE \texttt{Ordenar pasillos de mayor a menor}
\FOR{\texttt{pasillo in pasillos}}
	\IF {\texttt{El pasillo conecta dos conjuntos de pasillos que no ten\'ian conexión hasta el momento}}
		\STATE \texttt{Unir conjuntos conectados}
	\ELSE
		\STATE \texttt{Sumar la longitud del pasillo a la soluci\'on.}
	\ENDIF
\ENDFOR
\end{algorithmic} 


\subsection{Análisis de la complejidad}

\subsection{Código fuente}

\subsection{Experimentación}
\textcolor{blue}{Pensar casos borde}

\subsubsection{Constrastación Empírica de la complejidad}
