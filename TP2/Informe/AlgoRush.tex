\section{Problema 2: Algo Rush}

\subsection{Descripción de la problemática}
En este caso, se tiene un edificio con N pisos y con pasillos de longitud L y existen portales bidireccionales que comunican dos puntos determinados del mismo. Utilizar un portal consume dos segundos y caminar un metro consume un segundo. Se sabe que no hay más de un portal que comunique las mismas posiciones del mismo par de pisos.
Se requiere un algoritmo que determine la cantidad minima de segundos en ir desde el inicio del piso 0 hasta el final del último piso.

\subsection{Resolución propuesta y justificación}
Lo primero que hace el algoritmo es transformar la entrada en un grafo. Según nuestra representación, cada portal es considerado como un nodo. Las aristas que parten de un nodo llegan a todos aquellos otros nodos a los que se puede alcanzar desde el actual, ya sea porque son un punto de destino de teletransporte del mismo, o porque se encuentran en el mismo piso y se puede alcanzar caminando.

Como estructura auxiliar para ocnstruir el grafo se utiliza un diccionario de diccionarios de nodos, cuyas claves para acceder a un nodo son el piso y la distaancia respectivamente.
Un nodo contiene: un identificador (un int propio de cada nodo), la posicion en el pasillo donde se encuentra, un conjunto de nodos a los cuales se puede llegar caminando y un conjunto de nodos que son puntos de destino de teletransporte.

Posterirmente un segundo algoritmo, utilizando el grafo generado por el primero, se encarga de determinar el camino entre ambos extremos del grafo (desde el principio del piso 0 hasta el final del último piso). Para esto existe una función que calcula la distancia desde cualquier nodo hasta el extremo superior, la cual considera el camino minimo entre los nodos adyacentes y el extremo del último piso y, cuando el nodo buscado no estaba calculado anteriormente, se llama recursivamente a la función.

\subsection{Análisis de la complejidad}
Inicialmente, se debe construir un diccionario de N elementos y luego N diccionarios de L elementos, lo cual tiene un costo de O(nL). Luego se procesa la entrada que determina las posiciones de los portales y se van creando los nodos y clasificando en los diccionarios. Como la consulta y escritura en el diccionario es O(1), la complejidad total es de O(P). Finalmente, la complejidad de formar el grafo es de O(P+NL).
Para la busqueda del camino mínimo se crea inicialmente un diccionario de N*L elementos, lo cual tiene una complejidad de O(NL). Luego comienza a ejecutarse el algoritmo recursivo. Como el algoritmo nunca recorre dos veces el mismo nodo (ya que cada vez que resuelve el valor del camino mínimo para un nodo lo almacena en el diccionario) la complejidad es O(P).
Finalmente, el costo de construir el grafo y hallar el camino mínimo entre sus extremos es O(P+ NL).

\subsection{Código fuente}

\subsection{Experimentación}

\subsubsection{Constrastación Empírica de la complejidad}