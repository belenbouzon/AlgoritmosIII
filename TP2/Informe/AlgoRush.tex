\section{Problema 2: Algo Rush}

\subsection{Descripción de la problemática}
En este caso, se tiene un edificio con N pisos y con pasillos de longitud L y existen portales que comunican dos puntos determinados del mismo. Utilizar un portal consume dos segundos y caminar un metro consume un segundo. Se sabe que no existen dos portales a menos de un metro de distancia.
Se requiere un algoritmo que determine la cantidad minima de segundos en ir desde el inicio del piso 0 hasta el final del último piso.

\subsection{Resolución propuesta y justificación}
Según nuestra representación, cada portal es considerado como un nodo. Las aristas unen a todos aquellos a los que se puede alcanzar desde el actual, ya sea porque son un punto de destino de teletransporte del mismo, o porque se encuentran en el mismo piso y se puede alcanzar caminando.
Lo primero que hace el algoritmo es transformar la entrada en un grafo. Para eso genera temporalmente una estructura de diccionarios de diccionarios de nodos, los cuales cuyas claves son el piso y la distaancia respectivamente.
Un nodo contiene: un identificador (int), la posicion en el pasillo donde se encuentra, un conjunto de nodos a los cuales se puede llegar caminando y un conjunto de nodos que son puntos de destino de teletransporte

Posterirmente un segundo algoritmo se encarga de determinar el camino entre ambos extremos del grafo. Para esto existe una función que calcula la distancia desde cualquier nodo hasta el extremo superior, la cual prueba considerando el camino minimo con los nodos apuntados por cadauna de sus aristas y, cuando el nodo apuntado no estaba calculado anteriormente, se llama recursivamente.

\subsection{Análisis de la complejidad}
Inicialmente, se debe construir un diccionario de n elementos y luego n diccionarios de L elementos, lo cual tiene un costo de O(nL). Luego se recorre todo el string que determina las posiciones de los portales y se van creando los nodos y clasificando en los diccionarios. Como la consulta y escritura en el diccionario es O(1), la complejidad total es de O(p). Finalmente, la complejidad de formar el grafo es de O(p+nL).
Para la busqueda del camino mínimo se crea inicialmente un diccionario de n*l elementos, lo cual tiene una complejidad de O(nL). Luego comienza a ejecutarse el algoritmo recursivo. Como el algoritmo nunca recorre dos veces el mismo nodo (ya que cada vez que resuelve el valor delcamino mínimo para un nodo lo almacena en el diccionario) la complejidad es O(p)
Finalmente, el costo de construir el grafo y hallar el camino mínimo entre sus extremos es O(p+ nL).

\subsection{Código fuente}

\subsection{Experimentación}

\subsubsection{Constrastación Empírica de la complejidad}