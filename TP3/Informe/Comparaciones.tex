\section{Comparativas}

A continuación se exponen los gráficos generados a partir de los resultados de los experimentos que ideamos e implementamos con el fin de hacer un análisis empirico del costo temporal y la calidad de los resultados obtenidos a partir del algoritmo goloso (Ejercicio 3) y de la vecindad 1 del algoritmo de búsqueda local (Ejercicio 4). 

Para ello, separamos este apartado en la observación de tres conjuntos de resultados:

\begin{itemize}
\item {Los obtenidos a partir de grafos completos}
\item {Los obtenidos a partir de grafos cíclicos}
\item {Los obtenidos a parir de grafos bipartitos completos}
\end{itemize}

Consideramos que hacerlo fue de especial relevancia, dado que al incrementar individualmente el valor de parámetros independientes es factible que se terminen realizando experimentos sobre grafos estructuralmente distintos, con poca o ninguna relación en cuanto a sus propiedades.\\
Los resultadoss se observan a continuación:\\


\subsection {Resultados obtenidos a partir de grafos cíclicos} 

En las figuras \ref{TiempoGreedyCiclico} y \ref{TiempoLocalCiclico} muestran los tiempos medidos observados para los dinstintos algoritmos en grafos cíclicos de distintos tamaños. \\
Observándolos en su conjunto, se puede apreciar que no sólo el algoritmo goloso se comporta de manera más uniforme en función de la entrada, si no que también se hace notorio que el mismo demanda un tiempo de cómputo mucho mayor en cada uno de los casos.\\  
 
\begin{figure}[H]
    \begin{center}
  	\includegraphics[width=18cm]{imagenes/Ej5/TiempoGreedyCiclico.png}
 	\label{TiempoGreedyCiclico}
    \end{center}
  \end{figure}

 \begin{figure}[H]
    \begin{center}
  	\includegraphics[width=18cm]{imagenes/Ej5/TiempoLocalCiclico.png}
 	\label{TiempoLocalCiclico}
    \end{center}
  \end{figure}

Esto se manifiesta en la figura \ref{ComparacionTiemposCiclico}. Allí graficamos el porcentaje de tiempo ejecución de cada algoritmo para cada instancia en función del tiempo total de cómputo.\\
La comparativa se realizó de esta manera debido a que no son algoritmos que corran sobre instancias iguales: El algoritmo de búsqueda local busca mejorar un resultado ya obtenido por el algoritmo goloso y recibe nodos que se asumen bien coloreados de acuerdo a algún criterio prefijado.

 \begin{figure}[H]
    \begin{center}
  	\includegraphics[width=18cm]{imagenes/Ej5/ComparacionTiemposCiclico.png}
 	\label{ComparacionTiemposCiclico}
    \end{center}
  \end{figure}

Por último, el gráfico \ref{ComparacionConflictosCiclico} nos muestra que la cantidad de conflictos entre el output de los distintos algoritmos no varía notoriamente. Esto nos fuerza a concluir que en situaciones en las que se anticipa que el grafo pueda llegar a tener una estructura similar a uno cíclico y no sea estrictamente necesario conseguir un coloreo tan óptimo como sea posible, puede ser conveniente tomar los resultados arrojados por el primer algoritmo.

 \begin{figure}[H]
    \begin{center}
  	\includegraphics[width=18cm]{imagenes/Ej5/ComparacionConflictosCiclico.png}
 	\label{ComparacionConflictosCiclico}
    \end{center}
  \end{figure}


\subsection {Resultados obtenidos a partir de grafos completos} 

En las figuras \ref{TiempoGreedyCompleto} y \ref{TiemposLocalCompleto} muestran los tiempos medidos observados para 

 \begin{figure}[H]
    \begin{center}
  	\includegraphics[width=18cm]{imagenes/Ej5/TiempoGreedyCompleto.png}
 	\label{TiempoGreedyCompleto}
    \end{center}
  \end{figure}

 \begin{figure}[H]
    \begin{center}
  	\includegraphics[width=18cm]{imagenes/Ej5/TiemposLocalCompleto.png}
 	\label{TiemposLocalCompleto}
    \end{center}
  \end{figure}

 \begin{figure}[H]
    \begin{center}
  	\includegraphics[width=18cm]{imagenes/Ej5/ComparacionTiemposCompleto.png}
 	\label{ComparacionTiemposCompleto}
    \end{center}
  \end{figure}




 \begin{figure}[H]
    \begin{center}
  	\includegraphics[width=18cm]{imagenes/Ej5/ComparacionConflictosCompleto.png}
 	\label{ComparacionConflictosCompleto}
    \end{center}
  \end{figure}

\subsection {Resultados obtenidos a parir de grafos bipartitos completos} 