\newpage
\section{Problema 2: Algoritmo Exacto para List Coloring}

\subsection{Descripción de la problemática}
En esta oportunidad se nos pide resolver el problema List Coloring que consiste en (si es posible) colorear un grafo de forma que ningún nodo tenga el mismo color que un nodo adyacente, respetando la lista de colores de cada nodo, lo cual quiere decir que no se le puede asignar un color a un nodo si este no pertenecía a su lista de colores posibles.

\subsection{Resolución propuesta y justificación}

Para desarrollar un algoritmo exacto para este problema decidimos usar backtracking y de esta forma recorrer todas las opciones de colores y poder encontrar la solución si esta existe.\\

Como el enunciado pide que al llegar a un caso de 2-List Coloring se utilice el algoritmo del primer punto, decidimos cambiar el enfoque de la resolución para poder adaptarlo a instancias del ejercicio 1. Por lo tanto en lugar de usar la forma mas convencional de backtracking, que sería fijar un color para cada nodo y ver si es solución, optamos por recorrer todos los nodos del grafo fijando dos colores por cada uno y al llegar al último nodo llamar a 2-list Coloring para que chequee si con esa configuración de colores hay solución. Si la respuesta es $"X"$ entonces se seleccionan los dos colores siguientes del último nodo y se vuelve a probar. Cuando se acaban los colores del último, se elije los próximos colores del ante último y se vuelve a empezar de cero con los colores del siguiente y así hasta que se prueben todos los colores de todos los nodos. De esta forma siempre se llega a un caso base que es resoluble por 2-List Coloring.\\

Para evitar repetir o saltear colores se utiliza un iterador (\emph{it}) que siempre arranca desde el primer color de la lista y dentro del \texttt{while} se seleccionan dos colores consecutivos si hay o se selecciona uno solo (ver pseudocódigo de la sección \ref{sec:complj}) y luego se llama recursivamente a ListColoring para seleccionar los colores del nodo siguiente.\\

A continuación ilustramos la forma en la que se construye el árbol de soluciones para la siguiente lista de nodos:

\begin{center}
\emph{Grafo:} $\{ A, B, C, D \}$\\
Cada nodo tiene la siguiente cantidad de colores: $A: 2$, $B:4$, $C:5$ y $D:6$
\end{center}

\begin{center}
\begin{tikzpicture}[level 1/.style={level distance=1.5cm}]
\Tree
[.{A:0,1}
	[.{B:0,1}
		[.{C:0,1}
			[.{D:0,1} \texttt{2LC} ]
			[.{D:2,3} \texttt{2LC} ]
			[.{D:4,5} \texttt{2LC} ]
		]
		[.{C:2,3} {\vdots} ]
		[.{C:4} {\vdots} ]
	]
	[.{B:2,3}
		[.{C:0,1} {\vdots} ]
		[.{C:2,3} {\vdots} ]
		[.{C:4} {\vdots} ]
	]
]
\end{tikzpicture}
\end{center}

Como se puede observar en la imagen, cada nivel representa los llamados de \emph{ListColoring} para un determinado nodo y se puede ver como se seleccionan los índices de la lista de colores de a dos de forma que en las hojas se llama a \emph{2-List Coloring} (\texttt{2LC}) con todas las combinaciones de colores.

\newpage
\subsubsection{Podas}
\label{sec:podas}
\begin{enumerate}
	\item \label{it:p1} La primer poda que se realiza es la de ordenar la lista de nodos de forma que queden primero los nodos con menor cantidad de colores ya que de esta forma si se eligen los colores de un nodo $a$ con dos colores primero y luego se pasa a uno $b$ de cinco, se estaría recorriendo el subgrafo que no contiene a $a$ dos veces en lugar de las cinco que se recorrería si estuviera primero el nodo $b$.

	\item \label{it:p2} Al revisar el código de la primer entrega nos dimos cuenta que no era necesario hacer todas las permutaciones de los colores de un nodo ya que alcanzaba con seleccionar una sola vez cada color para cada nodo y llamar a 2-List Coloring dado que la recursión se encarga de que los colores se prueben todos con todos.

	\item \label{it:p3} Si se llega a una solución, se corta el backtracking y se devuelve esa. De esta forma en el caso promedio y mejor, no se recorrería todo el árbol de soluciones ya que no necesita encontrar todas las posibles.
	\item \label{it:p4} Por último se implementó una poda que preprocesa los datos y permite detectar si un grafo que tiene al menos un nodo con un solo color puede o no tener solución y en el caso que no pueda determinar efectivamente que no hay solución, igualmente poda todos los casos que generen conflictos con los nodos con un color. Para lograr esto, al momento de crear las estructuras de datos, se guardan en una cola fifo los índices de los nodos que tienen un solo color. Luego de que las estructuras están listas, se procede a analizar los vecinos de cada uno de los nodos de la cola, borrando -si existe- el único color del nodo de las listas de sus vecinos. Luego de borrar el color, se analiza el estado en el que quedó el vecino de la siguiente manera:
	\begin{itemize}
	\item Si le queda un solo color se agrega su índice al final de la cola.
	\item Si tiene un solo color y es el mismo del nodo que estamos analizando o le quedan cero colores, el algoritmo termina, pues no existe solución posible.
	\end{itemize}

\end{enumerate}

Por lo tanto como se está generando el árbol de soluciones completo y las podas cortan ramas solo cuando se encontró un coloreo posible o cuando se determina que no existe coloreo, se puede decir que nuestro algoritmo es correcto.

\subsection{Análisis de la complejidad}
\label{sec:complj}

Para analizar la complejidad utilizaremos el siguiente pseudocódigo, en el cual se utilizan los términos definidos a continuación:
\begin{itemize}
	\item \texttt{grafo}: para referirse al que viene por parámetro.

	\item \texttt{grafo2colores}: para el grafo que se construye para pasárselo al algoritmo del problema 1 y que admite hasta 2 colores por nodo.

	\item \texttt{ListColoring}: es el algoritmo que estamos analizando y representa el llamado recursivo a la misma función.

	\item \texttt{2ListColoring}: es el algoritmo del problema 1 y se usa para obtener la solución a un grafo con a lo sumo dos colores por nodo.

	\item $count$: es una variable pasada por parámetro que se utiliza para avanzar y retroceder en la lista de nodos.

	\item $c$: es la cantidad máxima de colores que puede tener un nodo y que está dada por la entrada del problema.
\end{itemize}

Como queremos analizar el peor caso, mostraremos las complejidades para el algoritmo sin podas, ya que de esta manera nos aseguramos de que siempre se recorra todo el árbol de soluciones del backtracking.
\newline

\begin{algorithm}[H]
\caption{ListColoring Exacto Sin Podas}
\label{lce}
\begin{algorithmic}[1]

\STATE $count = $ índice en la lista de nodos del \texttt{grafo} (inicializada en cero, viene por parámetro)

\IF {$count ==$ cantidad de nodos del \texttt{grafo}}
	\STATE Llamar a \texttt{2ListColoring} para resolver \texttt{grafo2colores} \COMMENT{$\mathcal{O}(n^2 \log{n})$} %complejidad del 1
	\STATE Retornar
\ENDIF

\STATE $nodo = $ Obtener el nodo a procesar del \texttt{grafo} con el índice $count$ \COMMENT{$\mathcal{O}(1)$} \label{get1}
\STATE $it = $ Obtener un iterador de la lista de colores de $nodo$ \COMMENT{$\mathcal{O}(1)$} \label{it}
\STATE $coloresSeleccionados = $ Obtener la lista de colores del $nodo$ en el \texttt{grafo2colores} \COMMENT{$\mathcal{O}(1)$} \label{get2}
\STATE Incrementar $count$ \COMMENT{$\mathcal{O}(1)$}

\WHILE[$\mathcal{O}(c/2)$ en cantidad de ejecuciones.]{Haya próximo en la lista de colores} \label{hn}
	\STATE $color1 = $ Próximo en la lista \COMMENT{$\mathcal{O}(1)$} \label{next}
	\STATE Asignar $color1$ en la primer posición de la lista $coloresSeleccionados$ \COMMENT{$\mathcal{O}(1)$}
	\IF{Hay próximo en la lista de colores}
		\STATE $color2 = $ Próximo en la lista \COMMENT{$\mathcal{O}(1)$} \label{nx}
		\STATE Asignar $color2$ en la segunda posición de la lista $coloresSeleccionados$ \COMMENT{$\mathcal{O}(1)$}
	\ENDIF
	\STATE Llamara a \texttt{ListColoring} con el $count$ ya incrementado
\ENDWHILE
\STATE Retornar
\end{algorithmic}
\end{algorithm}
\leavevmode
\newline
Por la documentación de \emph{ArrayList}\footnote{https://docs.oracle.com/javase/7/docs/api/java/util/ArrayList.html} sabemos que los métodos \texttt{get()} (lineas \ref{get1} y \ref{get2}), \texttt{listIterator()} (linea \ref{it}), \texttt{hasNext()} (linea \ref{hn}) y \texttt{next()} (lineas \ref{next} y \ref{nx}) corren en tiempo constante. Luego cada vez que se llega a una hoja del árbol de soluciones se llama a \texttt{2ListColoring} el cual tiene una complejidad de $\mathcal{O}(n^2 \log{n})$ por el análisis que se hizo en el problema 1. Cada nodo tiene como máximo $c$ colores, por lo tanto al seleccionar los colores de a dos, el ciclo termina teniendo una complejidad de $\mathcal{O}(c/2)$ ya que todo lo que se hace adentro es $\mathcal{O}(1)$. Por último se recorren recursivamente todos los nodos y por cada nodo se recorren $c/2$ colores, por lo tanto queda $\mathcal{O}((c/2)^n)$. En conclusión la complejidad sería:

\begin{center}
$(k*\mathcal{O}(1) + \mathcal{O}(c/2))^n * \mathcal{O}(n^2 \log{n}) = \mathcal{O}(c^n/2^n) * \mathcal{O}(n^2 \log{n})$ con $k$ constante.
\end{center}

\subsection{Experimentación}
Para hacer la experimentación generamos grafos al azar variando las aristas, los nodos y los colores\footnote{Siempre hace referencia a máximo de colores que puede tener un nodo, o sea, el tamaño del universo de colores en el cual están contenidas las listas de colores de cada nodo.} por separado para poder analizar como afecta cada una de estas componentes a la complejidad. Una vez generados los grafos, se corrió el algoritmo varias veces por cada uno, obteniendo los tiempos de ejecución, eliminando los outliers y calculando el promedio.\\

En la sección siguiente detallaremos la forma en la que se construyeron los grafos y en todos utilizaremos las leyendas:
\begin{itemize}
	\item \emph{Sin Podas}: El algoritmo se modificó para que esté obligado a recorrer todo el árbol de soluciones para cualquier entrada.

	\item \emph{Podas 1}: Hace referencia a las podas detalladas en los items \ref{it:p1}, \ref{it:p2} y \ref{it:p3} de la sección \ref{sec:podas} que se midieron todas juntas ya que la mas relevante es la que corta el backtracking cuando se encuentra una solución.

	\item \emph{Podas 2}: Hace referencia al item \ref{it:p4} de la sección \ref{sec:podas}, la cual se agregó a las otras podas y se midió nuevamente la performance para analizar si esta lograba mejoras ya que implementarla implica hacer un preprocesamiento de los datos.
\end{itemize}

\subsubsection{Aristas}

Lo primero que vamos a analizar son las aristas, para eso construimos grafos con cantidad de nodos ($n$) y de colores fijos, aumentando la cantidad de aristas hasta llegar al grafo completo de $n$ nodos.

\begin{figure}[H]
	\centering
 	\includegraphics[scale=0.45]{imagenes/Ej2/Aristas.png}
	\caption{Grafos de 10 nodos y 10 colores con variación de aristas de cinco en cinco.}
	\label{aristas}
 \end{figure}

 Como se puede ver en la figura \ref{aristas} los resultados sin podas tienen un comportamiento oscilante que no nos dice mucho sobre la complejidad, ya que eventualmente podrían tender a una constante, de hecho al intentar conseguir estos resultados, notamos que el algoritmo tardaba demasiado tiempo (en horas) para obtener las mediciones para los casos mas chicos y la única forma de conseguirlas en tiempo razonable fue bajando cantidad de nodos y la de colores. Por lo tanto creemos que este comportamiento se debe a que la cantidad de nodos cercana a tener el máximo de colores es mayor o menor respectivamente y esto condice el análisis de complejidad realizado ya que en el mismo no influye la cantidad de aristas.

\begin{figure}[H]
	\centering
 	\includegraphics[scale=0.45]{imagenes/Ej2/AristasPodas.png}
	\caption{Detalle ampliado del gráfico anterior para las podas 1 y 2.}
	\label{aristasP}
 \end{figure}

 Con respecto a las podas se puede ver que tienen un comportamiento mas constante, entre las podas 1 y 2 tiende a ser mejor la 2 (o sea, aplicar todas las podas juntas), pero las diferencias no son tan significativas al variar solo las aristas. Por otro lado, comparadas con las corridas sin podas, se puede observar que las mejoras también oscilan, en algunos casos las mejoras son enormes y en otros los tiempos se aproximan bastante y como las podas modifican la cantidad de nodos recorridos y/o los colores por nodo, llegamos a la misma conclusión que antes, que es que efectivamente las aristas no influyen en la complejidad temporal del algoritmo.\\

\subsubsection{Colores}

 Para analizar como influyen los colores se eligió una cantidad de nodos, las aristas del completo y para ese grafo se fue variando el máximo de colores por nodo

 \begin{figure}[H]
	\centering
 	\includegraphics[scale=0.45]{imagenes/Ej2/Colores.png}
	\caption{Grafos de 10 nodos, 45 aristas y colores que varían de 1 a 20.}
	\label{colores}
 \end{figure}

Al observar la figura \ref{colores} podemos ver que al principio, tanto los resultados sin podas como los de podas 1 y 2, son bastante similares y esto se debe a que hasta antes de 10 colores no existe solución ya que al ser grafos completos de 10 nodos, no se llega al mínimo de un color por nodo y la cantidad de colores a recorrer no es significativa para las mediciones sin podas. Luego de 10 y a partir del 12 vemos que los resultados se empiezan a diferenciar mas y volvemos a ver un comportamiento medio oscilante que como explicamos antes, se debe a que por mas que el máximo de colores sea uno, la cantidad real de colores que tiene cada nodo varía entre uno y el máximo representado por el eje $x$, por lo tanto cuantos mas nodos haya con cantidades de colores cercanas al máximo, mas tiempo va a tardar.

 \begin{figure}[H]
	\centering
 	\includegraphics[scale=0.45]{imagenes/Ej2/ColoresPodas.png}
	\caption{Detalle ampliado del gráfico anterior para las podas 1 y 2.}
	\label{coloresP}
 \end{figure}

Si miramos mas de cerca las podas en la figura \ref{coloresP} podemos observar que en algunos casos las mejoras fueron bastante considerables en \emph{Podas 2} y justamente coinciden con los casos en los que esta poda tiene mayor efecto. En particular el grafo con 12 colores tiene dos nodos con un único color y que es el mismo para los dos, por lo tanto a aplicar la poda \ref{it:p4} rápidamente se llega a ese problema y el algoritmo termina.\\

\subsubsection{Nodos}

Por último analizaremos el comportamiento al dejar fija la cantidad de colores, aumentar la cantidad de nodos y las aristas siempre son las del grafo completo para cada $n$. Notar que las aristas no se dejaron fijas ya que no tenía sentido (en especial para las podas) que a medida que aumentan las instancias sea mas fácil encontrar un coloreo porque el grafo es casi completamente disconexo.

 \begin{figure}[H]
	\centering
 	\includegraphics[scale=0.45]{imagenes/Ej2/Nodos.png}
	\caption{Grafos completos de 1 a 15 nodos con máximo 50 colores.}
	\label{nodos}
 \end{figure}

Al observar la figura \ref{nodos} vemos que efectivamente las mediciones de \emph{Sin Podas} muestran resultados exponenciales que escalan muy rápidamente con cantidades de nodos relativamente bajas\footnote{El algoritmo sin podas estuvo corriendo aproximadamente ocho horas y en ese tiempo no pudo obtener los resultados para 10 nodos, 45 aristas y 50 colores.}. Las podas a partir del grafo con 8 nodos muestran mejoras muy grandes ya que los resultados se mantienen mas o menos constantes mientras que el algoritmo sin podas crece exponencialmente.

 \begin{figure}[H]
	\centering
 	\includegraphics[scale=0.45]{imagenes/Ej2/NodosPodas.png}
	\caption{Detalle ampliado del gráfico anterior para las podas 1 y 2.}
	\label{nodosP}
 \end{figure}

 En la figura \ref{nodosP} se pueden ver unas variaciones en las que \emph{Podas 2} empeora un poco la performance de \emph{Podas 1} y eso se debe a que en esos puntos ya son grafos con cantidades de nodos mas considerables y al tener tantos colores disponibles es poco probable que haya nodos con un solo color, con lo cual esta poda no genera una diferencia positiva. A partir de ese momento los resultados con podas varían mucho si la solución se encuentra rápido o si se tiene que recorrer todo o casi todo el árbol, siendo los valores mas cercanos a cero o mas cercanos a los resultados exponenciales de \emph{Sin Podas}, respectivamente.\\

 En conclusión, luego de realizar este análisis podemos afirmar que los resultados sin podas condicen la cota de complejidad temporal propuesta y que al agregar las podas se pueden obtener resultados bastante mas rápido para valores pequeños de nodos y colores, pero indefectiblemente al aumentar estos valores el tiempo de cómputo se vuelve cada vez mas grande.
